\documentclass[12pt]{article}
 \usepackage{amsmath}
 \usepackage{latexsym}
 \usepackage{amsfonts}
 \usepackage[normalem]{ulem}
 \usepackage{soul}
 \usepackage{array}
 \usepackage{amssymb}
 \usepackage{extarrows}
 \usepackage{graphicx}
 \usepackage[backend=biber,
 style=numeric,
 sorting=none,
 isbn=false,
 doi=false,
 url=false,
 ]{biblatex}\addbibresource{bibliography.bib}
 
\usepackage{algorithm}
\usepackage{algpseudocode}
\usepackage{hyperref}
  \usepackage{subfig}
 \usepackage{wrapfig}
 \usepackage{txfonts}
 \usepackage{wasysym}
 \usepackage{enumitem}
 \usepackage{adjustbox}
 \usepackage{ragged2e}
 \usepackage[svgnames,table]{xcolor}
 \usepackage{tikz}
 \usepackage{longtable}
 \usepackage{changepage}
 \usepackage{setspace}
 \usepackage{hhline}
 \usepackage{multicol}
 \usepackage{tabto}
 \usepackage{float}
 \usepackage{multirow}
 \usepackage{makecell}
 \usepackage{fancyhdr}
 \usepackage[toc,page]{appendix}
 \usepackage[hidelinks]{hyperref}
 \usetikzlibrary{shapes.symbols,shapes.geometric,shadows,arrows.meta}
 \tikzset{>={Latex[width=1.5mm,length=2mm]}}
 \usepackage{flowchart}\usepackage[paperheight=11.69in,paperwidth=8.27in,left=1.18in,right=0.39in,top=0.79in,bottom=0.79in,headheight=1in]{geometry}
 \usepackage[utf8]{inputenc}
 \usepackage[russian]{babel}
 \usepackage[T1,T2A]{fontenc}
 \TabPositions{0.5in,1.0in,1.5in,2.0in,2.5in,3.0in,3.5in,4.0in,4.5in,5.0in,5.5in,6.0in,6.5in,}

  \urlstyle{same}

 
   %%%%%%%%%%%%  Set Depths for Sections  %%%%%%%%%%%%%%

  % 1) Section
 % 1.1) SubSection
 % 1.1.1) SubSubSection
 % 1.1.1.1) Paragraph
 % 1.1.1.1.1) Subparagraph

 
  \setcounter{tocdepth}{5}
 \setcounter{secnumdepth}{5}

 
   %%%%%%%%%%%%  Set Depths for Nested Lists created by \begin{enumerate}  %%%%%%%%%%%%%%

 
  \setlistdepth{9}
 \renewlist{enumerate}{enumerate}{9}
 		\setlist[enumerate,1]{label=\arabic*)}
 		\setlist[enumerate,2]{label=\alph*)}
 		\setlist[enumerate,3]{label=(\roman*)}
 		\setlist[enumerate,4]{label=(\arabic*)}
 		\setlist[enumerate,5]{label=(\Alph*)}
 		\setlist[enumerate,6]{label=(\Roman*)}
 		\setlist[enumerate,7]{label=\arabic*}
 		\setlist[enumerate,8]{label=\alph*}
 		\setlist[enumerate,9]{label=\roman*}

  \renewlist{itemize}{itemize}{9}
 		\setlist[itemize]{label=$\cdot$}
 		\setlist[itemize,1]{label=\textbullet}
 		\setlist[itemize,2]{label=$\circ$}
 		\setlist[itemize,3]{label=$\ast$}
 		\setlist[itemize,4]{label=$\dagger$}
 		\setlist[itemize,5]{label=$\triangleright$}
 		\setlist[itemize,6]{label=$\bigstar$}
 		\setlist[itemize,7]{label=$\blacklozenge$}
 		\setlist[itemize,8]{label=$\prime$}

  \pagenumbering{gobble}
 \setlength{\topsep}{0pt}\setlength{\parskip}{9.96pt}
 \setlength{\parindent}{0pt}

   %%%%%%%%%%%%  This sets linespacing (verticle gap between Lines) Default=1 %%%%%%%%%%%%%%

 
  \renewcommand{\arraystretch}{1.3}

 
  %%%%%%%%%%%%%%%%%%%% Document code starts here %%%%%%%%%%%%%%%%%%%%

 
 
  \begin{document}
 \begin{Center}
 ИТМО
 \end{Center}\par

  \begin{Center}
 Институт математики
 \end{Center}\par
 \vspace{\baselineskip}
 ОТЧЕТ \\
 ЗАЩИЩЕН С ОЦЕНКОЙ\par

  ПРЕПОДАВАТЕЛЬ\par

 
 
  %%%%%%%%%%%%%%%%%%%% Table No: 1 starts here %%%%%%%%%%%%%%%%%%%%

 
  \begin{table}[H]
  			\centering
 \begin{tabular}{p{2.05in}p{0.0in}p{1.76in}p{-0.01in}p{1.89in}}
 %row no:1
 \multicolumn{1}{p{2.05in}}{\Centering {проф., д.т.н., проф.}} & 
 \multicolumn{1}{p{0.0in}}{} & 
 \multicolumn{1}{p{1.76in}}{} & 
 \multicolumn{1}{p{-0.01in}}{} & 
 \multicolumn{1}{p{1.89in}}{\Centering {Прокопов Е.М.}} \\
 \hhline{-~-~-}
 %row no:2
 \multicolumn{1}{p{2.05in}} {\Centering{\fontsize{10pt}{12.0pt}\selectfont  {должность, уч. степень, звание}}} & 
 \multicolumn{1}{p{0.0in}}{} & 
 \multicolumn{1}{p{1.76in}} {\Centering{\fontsize{10pt}{12.0pt}\selectfont  {подпись, дата}}} & 
 \multicolumn{1}{p{-0.01in}}{} & 
 \multicolumn{1}{p{1.89in}} {\Centering{\fontsize{10pt}{12.0pt}\selectfont  {инициалы, фамилия}}} \\
 \hhline{~~~~~}

  \end{tabular}
  \end{table}

 
  %%%%%%%%%%%%%%%%%%%% Table No: 1 ends here %%%%%%%%%%%%%%%%%%%%

 
  \vspace{\baselineskip}

 
  %%%%%%%%%%%%%%%%%%%% Table No: 2 starts here %%%%%%%%%%%%%%%%%%%%

 
  \begin{table}[H]
  			\centering
 \begin{tabular}{p{6.49in}}
 %row no:1
 \multicolumn{1}{p{6.49in}}{\fontsize{14pt}{16.8pt}\selectfont {\section*{\Centering {ОТЧЕТ О ЛАБОРАТОРНОЙ РАБОТЕ №0}}}}\\
 \hhline{~}
 %row no:2
 \multicolumn{1}{p{6.49in}}{\section*{\Centering {РАБОТА С УДАЛЁННЫМИ РЕПОЗИТОРИЯМИ}}
 } \\
 \hhline{~}
 %row no:3
 \multicolumn{1}{p{6.49in}}{\subsubsection*{\Centering {по курсу: МАШИННОЕ ОБУЧЕНИЕ}}
 } \\
 \hhline{~}
 %row no:4
 \multicolumn{1}{p{6.49in}}{} \\
 \hhline{~}
 %row no:5
 \multicolumn{1}{p{6.49in}}{} \\
 \hhline{~}

  \end{tabular}
  \end{table}

 
  %%%%%%%%%%%%%%%%%%%% Table No: 2 ends here %%%%%%%%%%%%%%%%%%%%

  РАБОТУ ВЫПОЛНИЛ\par

 
 
  %%%%%%%%%%%%%%%%%%%% Table No: 3 starts here %%%%%%%%%%%%%%%%%%%%

 
  \begin{table}[H]
  			\centering
 \begin{tabular}{p{1.3in}p{1.0in}p{-0.04in}p{1.63in}p{-0.04in}p{1.63in}}
 %row no:1
 \multicolumn{1}{p{1.3in}}{СТУДЕНТ ГР. №} & 
 \multicolumn{1}{p{1.0in}}{\Centering {R3335}} & 
 \multicolumn{1}{p{-0.04in}}{} & 
 \multicolumn{1}{p{1.63in}}{} & 
 \multicolumn{1}{p{-0.04in}}{} & 
 \multicolumn{1}{p{1.63in}}{\Centering {Д.М. Свердлов}} \\
 \hhline{~-~-~-}
 %row no:2
 \multicolumn{1}{p{1.3in}}{} & 
 \multicolumn{1}{p{1.0in}}{} & 
 \multicolumn{1}{p{-0.04in}}{} & 
 \multicolumn{1}{p{1.63in}} {\Centering{\fontsize{10pt}{12.0pt}\selectfont  {подпись, дата}}} & 
 \multicolumn{1}{p{-0.04in}}{} & 
 \multicolumn{1}{p{1.63in}} {\Centering{\fontsize{10pt}{12.0pt}\selectfont  {инициалы, фамилия}}} \\
 \hhline{~~~~~~}

  \end{tabular}
  \end{table}

 
  %%%%%%%%%%%%%%%%%%%% Table No: 3 ends here %%%%%%%%%%%%%%%%%%%%

 
  \vspace{\baselineskip}
 \vspace{\baselineskip}
 \vspace{\baselineskip}
 \vspace{\baselineskip}
 \vspace{\baselineskip}
 \usepackage{verbatim}
 \begin{Center}
 Санкт-Петербург \the\year{}
 \end{Center}\par
 \newpage


\section*{Задание 1}
При выполнение задания 1, необходимо было сощдать файлы в удалённом репозитории и наполнить его файлами, чтобы позже рабоать с ними в локальном репозитории. Рассмотрим содержимое созданных файлов:

\subsection{Содержимое файла goals.md}
\begin{algorithm}[H]
\caption{Содержимое файла goals.md}
\begin{algorithmic}[1]
\State \textbf{\# Мой опыт}
\State На первом курсе, с командой из 3-х человек написали тг-бота.
\State Личный проект - голосовой помощник на базе MacOS.
\State Ознакомиться можно: (здесь)[\url{https://github.com/GeraDrobierro}]
\State В личных интересах работал над созданием полиномиального и матричного калькулятора,
\State однако его нигде не размещал.
\State
\State \textbf{\# Мои цели}
\State \textbf{1.} Начать изучение теории алгоритмов
\State \quad + изучить графы
\State \quad + Изучить асимптотику
\State \textbf{2.} Подтянуть знания в машинном обучении
\State \quad + ознакомиться с задачами машинного обучения
\State
\State \textbf{Вставка изображения:} ![фото](resources/кот.jpeg)
\end{algorithmic}
\end{algorithm}

\subsection{Содержимое файла info.md}
\begin{algorithm}[H]
\caption{Содержимое файла info.md}
\begin{algorithmic}[1]
\State \textbf{ФИО:} Свердлов Давид Михайлович
\State \textbf{ОС:} macOS
\State \textbf{Дата:} 2025-09-09 09:21
\end{algorithmic}
\end{algorithm}
\subsection{Содержимое файла .gitignore}
\begin{algorithm}[H]
\caption{Содержимое файла .gitignore}
\begin{algorithmic}[1]
\State data/
\end{algorithmic}
\end{algorithm}

\section*{Задание 2}
\subsection{Объяснение результата попытки закоммитить файл data/dataset.csv}
Игнорируемые файлы отслеживаются в специальном файле .gitignore, . В Git  для указания игнорируемых файлов нужно задать .gitignore, чтобы указать в нем новые файлы, которые должны быть проигнорированы. Файлы .gitignore.


При выполнении команды \verb|git add data/dataset.csv| не происходит никаких видимых действий -- файл не добавляется в отслеживаемые. Команда \verb|git status|  показывает, что файл \verb|data/dataset.csv| является неотслеживаемым файлом и при этом  любые изменения внутри этого файла игнорируются системой контроля версий\\

Следовательно, последующая команда \verb|git commit| не включает файл \verb|data/dataset.csv| в коммит. В удалённый репозиторий на GitHub отправляется только коммит с файлом \verb|.gitignore|. Сама папка \verb|data| и файл \verb|dataset.csv| остаются только в локальной рабочей копии и не попадают в историю версий.





\end{document}
